\chapter{Technical Info}

\section{Installation}
\label{sec:installation}

\subsection{Requirements}

The Pipeline tools are available to all employees of ICE Animations who have
access to company intranet on the domain ICEANIMATIONS.

The User needs to mount his "R:" to the location 
\begin{quote}
$\backslash$$\backslash$ice-sql$\backslash$Storage$\backslash$repository
\end{quote}

This is automatically done for all domain users using active directory user
login policy.


\subsubsection{Supported Platforms} 

At the time of writing this document, the ICE Animations pipeline Toolbox has
been tested only for the Windows OS and Maya 2014 platforms.


\subsection{Remote Installation}

This mode of installation requires administrator privileges. Installation of
the Pipeline Toolbox can now be done remotely by running the following file
using an account which has administrator status on all intended machines. One
such account is \emph{ICEANIMATIONS$\backslash$Administrator}

\begin{quote}
R:$\backslash$Pipe\_Repo$\backslash$Projects$\backslash$TACTIC$\backslash$client\_batch$\backslash$run\_remote\_deployer.bat
\end{quote}

\subsection{User Installation}

The ICE Menu inside Maya can be installed by copying the file named
\emph{userSetup.mel} from the following location 
\begin{quote}
R:$\backslash$Pipe\_Repo$\backslash$Users$\backslash$Hussain$\backslash$utilities$\backslash$loader$\backslash$client
\end{quote}
into the users \emph{maya/scripts} folder or the
\emph{maya/$<$MAYAVERSION$>$/scripts} in the \emph{My Documents} directory

If the ICE Menu does not appear, remove any other files with the name
\emph{userSetup.mel} from all the locations in the env variable
\emph{\%MAYA\_SCRIPT\_PATH\%}

The ICE Menu in Program Folders can be added by adding a shortcut to the
start menu to the folder

\begin{quote}
R:$\backslash$Pipe\_Repo$\backslash$Users$\backslash$Qurban$\backslash$applications
\end{quote}

This makes available all the batch files which can be used to run a particular
standalone tool in the toolbox.


\section{Distribution Mechanism}

Before distribution of the tools; the first question is whether the tool will
run from within Maya or as an independent desktop application. If the tool is
required to run from within Maya, then the tool is deployed as item in
drop-down menu in the main menu of Maya under “ICE Scripts” menu, so the user
has an easy access to the tools when needed.

In order to create “ICE Scripts” menu in Maya, a “Mel” script in needed which
is executed each time the Maya is launched by the user. This script reads a
predefined directory from a known location in the file system and creates a
menu named “ICE Scripts” containing the names of the sub-directories within
it. Names of these sub-directories represent the departments in the company.
Each sub-directory contains tools (Mel and/or Python scripts) which belong to
the department sub-directory represents. These tools are either based on a
single file or single file contains the script to import and run a complete
software package from a specified directory. The location for the directory
which contain the scripts is R:/mel\_scripts and the location for the directory
which contains complete software packages is R:/Python\_Scripts/plugins.

The script which create “ICE Scripts” menu in Maya can be found in
R:/Pipe\_Repo/Users/Hussain/utilities/loader/client; as userSetup.mel. You can
copy and paste this script to your Maya installation directory (C:/Program
Files/Autodesk/Maya2011/scripts/startup) to have the “ICE Scripts” menu in
Maya or you can create a shortcut to the file in startup directory.

If the tool is required to run as a desktop application, the source of the
tool/software is placed in directory accessible to all of the users in the
company and a batch file is created to run a tool. This batch file is made
available to all of the users so that they can run the required tool from
their local machine. 

