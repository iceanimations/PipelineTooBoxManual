\documentclass[10pt,a4paper,openany,oneside,onecolumn,titlepage]{report}
\usepackage[utf8]{inputenc}
\usepackage{amsmath}
\usepackage{amsfonts}
\usepackage{amssymb}
\usepackage{graphicx}
\usepackage[bookmarks]{hyperref}
\author{
        Talha Ahmed \\ Computer Graphics Researcher \and 
        Qurban Ali \\ CGI Pipeline Engineer
}
\title{ Pipeline Toolbox \\ ICE Animations }
\date{\today}
\setcounter{secnumdepth}{1}
\setcounter{tocdepth}{1}
\begin{document}
\maketitle
\tableofcontents \newpage
\chapter{Introduction}
This document describes the tools which developed and deployed by the Pipeline
Team at ICE Animations over the years.

The purpose of this document is to present a description of different software
available to the end user from the production team. This document can also be
used as a reference for learning what each tool in the menu does.

These pipeline tools have been developed and accumulated over the years by
different individuals since the inception of many projects. They have catered
to the workflow needs of many individuals working in many production projects.
We have limited amount of data as to how many times each tool has been used
during production. In this context, the evolution of this tool box in the near
future at least is heavily dependent on the feedback and requirements
communication apart from the capability of the development team. This document
should help the enduser of our pipeline tools to get an idea of what kind of
automations can be done, in order to suggest more ideas and improvements

The Pipeline team has produced and deployed broadly two different kind of
software for the production team. The first kind of software is what runs
inside Maya and other production software. The second kind of software runs in
the windows OS as standalone software. Both these kind of software have their
utility in different stages of the production pipeline.

The Pipeline team is responsible for maintaining and updating the tool box,
and our IT team takes care that all the software and data contained
within these toolboxes is backed up properly according to the needs of the
pipeline.

\chapter{Technical Info}

\section{Installation}
\label{sec:installation}

\subsection{Requirements}

The Pipeline tools are available to all employees of ICE Animations who have
access to company intranet on the domain ICEANIMATIONS.

The User needs to mount his "R:" to the location 
\begin{quote}
$\backslash$$\backslash$ice-sql$\backslash$Storage$\backslash$repository
\end{quote}

This is automatically done for all domain users using active directory user
login policy.


\subsubsection{Supported Platforms} 

At the time of writing this document, the ICE Animations pipeline Toolbox has
been tested only for the Windows OS and Maya 2014 platforms.


\subsection{Remote Installation}

This mode of installation requires administrator privileges. Installation of
the Pipeline Toolbox can now be done remotely by running the following file
using an account which has administrator status on all intended machines. One
such account is \emph{ICEANIMATIONS$\backslash$Administrator}

\begin{quote}
R:$\backslash$Pipe\_Repo$\backslash$Projects$\backslash$TACTIC$\backslash$client\_batch$\backslash$run\_remote\_deployer.bat
\end{quote}

\subsection{User Installation}

The ICE Menu inside Maya can be installed by copying the file named
\emph{userSetup.mel} from the following location 
\begin{quote}
R:$\backslash$Pipe\_Repo$\backslash$Users$\backslash$Hussain$\backslash$utilities$\backslash$loader$\backslash$client
\end{quote}
into the users \emph{maya/scripts} folder or the
\emph{maya/$<$MAYAVERSION$>$/scripts} in the \emph{My Documents} directory

If the ICE Menu does not appear, remove any other files with the name
\emph{userSetup.mel} from all the locations in the env variable
\emph{\%MAYA\_SCRIPT\_PATH\%}

The ICE Menu in Program Folders can be added by adding a shortcut to the
start menu to the folder

\begin{quote}
R:$\backslash$Pipe\_Repo$\backslash$Users$\backslash$Qurban$\backslash$applications
\end{quote}

This makes available all the batch files which can be used to run a particular
standalone tool in the toolbox.


\section{Distribution Mechanism}

Before distribution of the tools; the first question is whether the tool will
run from within Maya or as an independent desktop application. If the tool is
required to run from within Maya, then the tool is deployed as item in
drop-down menu in the main menu of Maya under “ICE Scripts” menu, so the user
has an easy access to the tools when needed.

In order to create “ICE Scripts” menu in Maya, a “Mel” script in needed which
is executed each time the Maya is launched by the user. This script reads a
predefined directory from a known location in the file system and creates a
menu named “ICE Scripts” containing the names of the sub-directories within
it. Names of these sub-directories represent the departments in the company.
Each sub-directory contains tools (Mel and/or Python scripts) which belong to
the department sub-directory represents. These tools are either based on a
single file or single file contains the script to import and run a complete
software package from a specified directory. The location for the directory
which contain the scripts is R:/mel\_scripts and the location for the directory
which contains complete software packages is R:/Python\_Scripts/plugins.

The script which create “ICE Scripts” menu in Maya can be found in
R:/Pipe\_Repo/Users/Hussain/utilities/loader/client; as userSetup.mel. You can
copy and paste this script to your Maya installation directory (C:/Program
Files/Autodesk/Maya2011/scripts/startup) to have the “ICE Scripts” menu in
Maya or you can create a shortcut to the file in startup directory.

If the tool is required to run as a desktop application, the source of the
tool/software is placed in directory accessible to all of the users in the
company and a batch file is created to run a tool. This batch file is made
available to all of the users so that they can run the required tool from
their local machine. 



\chapter{ICE Menu - Toolbox Inside Autodesk Maya}
This Chapter describes the different custom tools available in Autodesk Maya
through a program menu. In order to install this menu when on the
ICEANIMATIONS intranet refer to the installation section
\ref{sec:installation}
All the available tools have been classified roughly according to the
department or pipeline stage they are most likely required in.

\section{Animation Toolbox}
This section describes the tools included in the animation toolbox section


\chapter{ICE Programs Menu}

This Chapter describes programs in the Pipeline Toolbox that run in the
windows OS outside but not necessarily independant of the production software.

\section{ExpoCache}

“Expo Cache” was designed and developed to enable the users to export caches
from the Scene files without opening them. This tool lists all the sets form a
scene file and then exports the caches of the selected sets to the location
specified by the user.

\subsection*{Features}

\begin{itemize}
\item Provides option to change the number of frames per second
\item Provide the option to select the camera from the scene (if number of cams is more than 1)
\item Directly exports camera from the scene (if there is only one cam)
\end{itemize}

\subsection*{Importance/Usage}

Expo Cache got 11 hits during the month of August 2013

\subsection*{Location/Access}

\subsubsection*{Source code}
R:$\backslash$Pipe\_Repo$\backslash$Users$\backslash$Qurban$\backslash$src (expoCache)

\subsubsection*{Runable}
R:$\backslash$Pipe\_Repo$\backslash$Users$\backslash$Qurban$\backslash$applications$\backslash$runables$\backslash$expoCache (expoCache.bat)

\subsection*{How to run} 
Copy .bat file to your desktop or create shortcut to your desktop, double
click to run.



\section{ExpoFast}

Originally developed as “Expo Cam” to export the camera from the scenes, later
the options to export the shaders and light were also provided by the tool so
I named it as “Expo Fast” as it quickly opens the scene files and exports the
objects from the scene. User can select the object type to be exported the
tool will list the objects and then click on the export button the objects
will be exported to the specified location. User can export objects as
separate files or he/she can export them as single file.

\subsection*{Features}
\begin{itemize}
\item Provides option to select the objects (camera, shader, light)
\item Provide option to export the objects as separate files
\item Provide option to export objects as a single file
\end{itemize}

\subsection*{Importance/Usage}

“Expo Fast” got 10 hits during the month of August 2013

\subsection*{Location/Access}

\subsubsection{Source code} 

R:$\backslash$Pipe\_Repo$\backslash$Users$\backslash$Qurban$\backslash$src (expoFast)

\subsubsection{Runable} 

R:$\backslash$Pipe$\backslash$\_Repo$\backslash$Users$\backslash$Qurban$\backslash$applications$\backslash$runables$\backslash$expoFast (expoFast.bat)

\subsubsection{How to run}

Copy .bat file to your desktop or create shortcut to your desktop, double
click to run.


\section{FrameChecker}

This tool lists all the render frames, lists the missing frames and bad
frames. Missing frames are ones which are not rendered and the folder/director
does not contain those frames. Bad frames are ones which are rendered but not
properly. The criterion for bad frames is small size as compared to size of
all the frames. The tool just notifies the user by listing bad frames that
these frames have a very large variance in size as compared to other frames.
Usually bad frames have a very small size. This indicates that these frames
might not be properly rendered.

\subsection*{Features}

\begin{itemize}
\item Lists missing frames
\item Lists bad frames
\item Makes the bad frame names hyperlinks to the actual files in  the file
system
\item User can right click on any of the frames and choose “check render
frames” option from the context menu that appears to launch the “Frames
Checker”.
\end{itemize}

\subsection*{Importance/Usage}

“Frame Checker” got 100 hits during the month of August 2013

\subsection*{Location/Access}

\subsubsection*{Source code} 

R:$\backslash$Pipe$\backslash$\_Repo$\backslash$Users$\backslash$Qurban$\backslash$src (frameChecker)

\subsubsection*{Runable} 

R:$\backslash$Pipe\_Repo$\backslash$Users$\backslash$Qurban$\backslash$applications$\backslash$runables$\backslash$frameChecker (frameChecker.bat)

\subsubsection*{How to run}  

Copy .bat file to your desktop or create shortcut to your desktop, double click it. You can also right click the any frame and select “check render frames” option (if you don’t see this option, please contact to Pipeline team).


\section{PaintDesk}

“Paint Desk” is designed to assist the users to make the notes on the current
desktop state by painting it with a brush or pen. User can save these notes to
predefined location or he/she can select where to save at time the painting is
completed.

\subsection*{Features}
\begin{itemize}
\item Allow the user to paint the current state of the desktop
\item Allow the user to save these painted images to a predefined location
\item Allow the user to specify the location at the run time
\item Allow the user to change the preferences at run time
\item General options for painting include erasing, undo, redo, change pen color, clear
\item Changing size of pen and eraser by scroll button of the mouse
\end{itemize}

\subsection*{Importance/Usage}

“Paint Desk” got 7 hits during first 10 days
2 people are familiar with this tool

\subsection*{Location/Access}
\subsubsection*{Source code}

R:$\backslash$Pipe\_Repo$\backslash$Users$\backslash$Qurban$\backslash$src (paintDesk)

\subsubsection*{Runable} 

R:$\backslash$Pipe\_Repo$\backslash$Users$\backslash$Qurban$\backslash$applications$\backslash$runables$\backslash$paintDesk (paintDesk.bat)

\subsubsection*{How to run} 

To run the tool create a shortcut to your desktop and assign a shortcut key to
it by right clicking on the icon, go to the properties > shortcut > shortcut
key. Press the keys you want to assign.



\chapter{Conclusion}

The pipeline team at ICE Animations like in any other studio is essentially a
support center created to serve the production team in order to help them
organize and speed up their work through automation. The advantage and value
of an in-house developed tool is its conciseness and ability to tune it
towards the exact needs of the production team or individuals’ workflow. In
this context, the progress of the pipeline team directly depends upon the
quality and the quantity of ideas generated by individuals in the production
team and those communicated aptly to the development team. Also, the metrics
which define separate success from failure and the ones which measure progress
for this team are also determinable essentially from its contribution in
production.

A scan through of this document should enable the reader to gage
the kind of tools needed in our production pipeline, along with the capability
of the development team and also provide the basis for more creative thinking
which may lead towards ideas for development of more tools. The pipeline team
welcomes all such ideas for discussion and possible implementation.


\end{document}
